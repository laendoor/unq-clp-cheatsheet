\documentclass[twoside,a4paper,12pt]{article}
\usepackage{color,soul}
\usepackage{cmll}
\usepackage{ulem}
\usepackage[spanish,es-nodecimaldot]{babel}
\usepackage{fancyhdr}
\usepackage{amssymb,amsmath,amsthm,bm}
\usepackage[llbracket,rrbracket]{stmaryrd}
\usepackage[top=1in, bottom=1.5in, left=1in, right=1in]{geometry}
\usepackage[T1]{fontenc}
\usepackage[utf8]{inputenc}
\usepackage{multicol}
\usepackage{url}
\usepackage{proof}
\usepackage{cancel}
\usepackage{tikz}
\usetikzlibrary{babel,arrows}
\usepackage{nicefrac}
\usepackage{adjustbox,varwidth,xparse}
\usepackage{longtable}
\usepackage[square]{natbib}
\usepackage{hyperref}
\hypersetup{pdfpagemode=UseOutlines}
\hypersetup{colorlinks=true}
\hypersetup{urlcolor=red}
\hypersetup{linkcolor=black}
\hypersetup{citecolor=black}
\usepackage{multirow}
\allowdisplaybreaks


\NewDocumentEnvironment{bracedrows}{m}
  {\begin{adjustbox}{valign=t}%
   $\kern-\nulldelimiterspace\left.
   \begin{tabular}{@{}l@{}}}
  {\end{tabular}\right\rbrace
   \begin{varwidth}{.5\linewidth}#1\end{varwidth}$%
   \end{adjustbox}}


\parindent 0pt
\let\tempone\itemize
\let\temptwo\enditemize
\renewenvironment{itemize}{\tempone\addtolength{\itemsep}{-0.4\baselineskip}}{\temptwo}

\theoremstyle{definition}

\newtheorem{innercustomgeneric}{\customgenericname}
\providecommand{\customgenericname}{}
\newcommand{\newcustomtheorem}[2]{%
  \newenvironment{#1}[1]
  {%
   \renewcommand\customgenericname{#2}%
   \renewcommand\theinnercustomgeneric{##1}%
   \innercustomgeneric
  }
  {\endinnercustomgeneric}
}

\newcustomtheorem{definicion}{Definición}
\newcustomtheorem{teorema}{Teorema}
\newcustomtheorem{lema}{Lema}

% \newtheorem{definicion}{Definición}[section]
% \newtheorem{teorema}[definicion]{Teorema}
% \newtheorem{lema}[definicion]{Lema}
\newtheorem*{propiedades}{Propiedades}
\newtheorem*{propiedad}{Propiedad}
\theoremstyle{remark}
\newtheorem*{observacion}{Observación}
\newtheorem*{observaciones}{Observaciones}
% \newtheorem{ejemplo}[definicion]{Ejemplo}
% \newtheorem{ejemplos}[definicion]{Ejemplos}

% \pagestyle{fancy}
% \fancyhf{}
% \fancyhead[RO,LE]{\thepage}
% \fancyhead[LO]{\rmfamily\sc\nouppercase\leftmark}
% \fancyhead[RE]{\rmfamily\sc\nouppercase\rightmark}
% \setlength{\headheight}{27.16pt}

% Comandos
% Comandos de Jano + propios (mantener orden alfabético)
\newcommand{\A}[2]{\ensuremath{\forall {#1}{.}{#2}}}
\newcommand{\append}{:}
\newcommand{\nil}{\mathsf{nil}}
\newcommand{\bool}{\ensuremath{\mathsf{bool}}}
\newcommand{\coma}{;}
\newcommand{\cons}[2]{\mathsf{cons~}{#1}{#2}}
\newcommand{\hd}[1]{\mathsf{hd~}{#1}}
\newcommand{\tl}[1]{\mathsf{tl~}{#1}}
\newcommand{\concat}{\!+\!\!\!\!+}
\newcommand{\concats}{\textrm{\tiny $+\!\!\!+$}}
\newcommand{\error}{\mathtt{error}}
\newcommand{\false}{\ensuremath{\mathsf{false}}}
\newcommand{\fix}[2]{\ensuremath{\mathsf{\mu}#1.#2}}
\newcommand{\fixfun}[3]{\ensuremath{\mathsf{fixfun~}{#1}{\ }{#2}{.}{#3}}}
\newcommand{\FN}{\ensuremath{\mathsf{FN}}}
\newcommand{\fst}{\ensuremath{\mathsf{fst}}}
\newcommand{\fun}[2]{\lambda {#1}{.}{#2}}
\newcommand{\funSem}[3]{{#1}^{\in \sem{#2}} \mapsto{#3}}
\newcommand{\FV}{\ensuremath{\mathsf{FV}}}
\newcommand{\Gen}{\ensuremath{\mathsf{Gen}}}
\newcommand{\hra}{\hookrightarrow}
\newcommand{\ife}[3]{\ensuremath{\mathsf{if~}#1\mathsf{~then~}#2\mathsf{~else~}#3}}
\newcommand{\ifnil}[3]{\ensuremath{\mathsf{ifnil~}#1\mathsf{~then~}#2\mathsf{~else~}#3}}
\newcommand{\ifz}[3]{\ensuremath{\mathsf{ifz~}#1\mathsf{~then~}#2\mathsf{~else~}#3}}
\newcommand{\letl}[3]{\mathsf{let~}#1=#2\mathsf{~in~}#3}
\newcommand{\listl}[1]{\ensuremath{\mathsf{list[#1]}}}
\newcommand{\mgu}{\ensuremath{\mathsf{mgu}}}
\newcommand{\nat}{\mathsf{nat}}
\newcommand{\One}{\mathbf 1}
\newcommand{\pair}[1]{\ensuremath{\mathsf{Pair}~#1}}
\newcommand{\s}[1]{\ensuremath{\mathsf{#1}}}
\newcommand{\self}{\ensuremath{\mathsf{self}}}
\newcommand{\sem}[1]{\ensuremath{\llbracket{#1}\rrbracket}}
\newcommand{\semT}[2][\theta]{\ensuremath{\sem{#2}_{#1}}}
\newcommand{\set}[1]{\ensuremath{\{#1\}}}
\newcommand{\snd}{\ensuremath{\mathsf{snd}}}
\newcommand{\then}{\Rightarrow}
\newcommand{\thunk}[2]{\langle{#1},{#2}\rangle}
\newcommand{\true}{\ensuremath{\mathsf{true}}}


\hypersetup{
  urlcolor=blue,
}

% \title{Características de lenguajes de programación}
% \author{Alejandro Díaz-Caro \& Rafael Romero}
% \date{2020}

\begin{document}
\begin{center}
  {\Large UNQ | CLP | Cheat Sheet}

  {\small Basado en 
    \href
    {http://clp.web.unq.edu.ar/wp-content/uploads/sites/110/2021/08/apuntes.pdf}
    {Apuntes de la Materia}
  }
\end{center}

\section*{PCF no tipado}

\subsection*{Gramática}

\begin{equation*}
  t ::= x~|~\fun xt~|~tt~|~n\in\mathbb N~|~t+t~|~t-t~|~t\times t~|~t/t
         ~|~\ifz ttt~|~\fix xt~|~\letl xtt
\end{equation*}

\subsection*{Semántica Operacional}

\subsubsection*{Reglas Base}

\begin{align}
  (\fun xu)t &\to u[t/x] & \textrm{($\beta$ reducción)}\\
  p\otimes q &\to n &\textrm{ Si $p\otimes q=n$, con $\otimes=+,-,\times$ o $/$}\\
  \ifz 0tu &\to t\\
  \ifz ntu &\to u &\textrm{ Si $n\neq0$ }\\
  \fix xt &\to t[\fix xt/x]\\
  \letl xtu &\to u[t/x]
\end{align}

\subsubsection*{Reglas de Congruencia}

\[\arraycolsep=3pt\def\arraystretch{2.6}
  \begin{array}{c c c}
      \infer[rc_1]{tv\to uv}{t\to u}
      & \infer[rc_2]{vt\to vu}{t\to u}
      &\infer[rc_3]{\fun xt\to\fun xu}{t\to u} \\
      \infer[rc_4]{t\otimes v\to u\otimes v}{t\to u}
      & \infer[rc_5]{v\otimes t\to v\otimes u}{t\to u}
      &\infer[rc_6]{\ifz tsv \to \ifz usv}{t\to u} \\
      \infer[rc_7]{\fix ft \to \fix fu}{t\to u}
      & \infer[rc_8]{\letl xts \to \letl xus}{t\to u}
      &\infer[rc_9]{\letl xst \to \letl xsu}{t\to u}
  \end{array}
\]

\newpage
\subsubsection*{Definición de $t[u/x]$ (por inducción en $t$)}

\begin{align*}
  x[u/x] &= u\\
  y[u/x] &= y\\
  (\fun xt)[u/x] &=\fun xt\\
  (\fun yt)[u/x] &=\fun y{t[u/x]} &\textrm{Si }y\notin \FV(u)\\
  (\fun yt)[u/x] &=\fun z{t[z/y]}[u/x] &\textrm{Si }y\in \FV(u)\\
  (tv)[u/x] &= t[u/x]v[u/x]\\
  n[u/x] &=n\\
  (t\otimes v)[u/x] &=t[u/x]\otimes v[u/x]\\
  (\ifz t{v_1}{v_2})[u/x] &=\ifz{t[u/x]}{v_1[u/x]}{v_2[u/x]}\\
  (\fix xt)[u/x] &=\fix xt\\
  (\fix yt)[u/x] &=\fix yt[u/x] &\textrm{Si } y\notin \FV(u)\\
  (\fix yt)[u/x] &=\fix zt[z/y][u/x] &\textrm{Si }y\in \FV(u)\\
  (\letl xtv)[u/x] &=\letl x{t[u/x]}v\\
  (\letl ytv)[u/x] &=\letl y{t[u/x]}{v[u/x]} &\textrm{Si }y\notin \FV(u)\\
  (\letl ytv)[u/x] &=\letl y{t[u/x]}{v[z/y][u/x]} &\textrm{Si }y\in \FV(u)
\end{align*}

\newpage
\section*{Estrategias de Reducción}

\begin{definicion}{3.1}
  Notamos $\to^*$ al cierre reflexivo y transitivo de $\to$.
\end{definicion}

\begin{definicion}{3.3}
  Un término $t$:
  \begin{enumerate}
  \item está en forma normal si $\nexists$ $u$ tal que $t \to u$.
  \item es normalizable si $\exists$ $u$ en forma normal tal que $t \to^*u$.
  \item es fuertemente normalizable si $\nexists$ secuencia
    infinita $v_0$, $v_1$, $\dots$ $|$ $t\to v_0\to v_1\to\dots$.
  \end{enumerate}
\end{definicion}

\begin{definicion}{3.4}
  Sea $\to_R$ una relación binaria, y $\to^*_R$ su cierre reflexivo y transitivo.
  \begin{itemize}
  \item
    \begin{tikzpicture}[baseline=-1.5ex]
      \node at (-7,-.4) {\parbox{0.7\textwidth}{ $\to_R$ satisface la
          \emph{propiedad del diamante} si $t\to_R v_1$ y $t\to_R v_2$ implica
          que $v_1\to_R u$ y $v_2\to_R u$ para algún $u$.} }; \node (t) at (0,0)
      {$t$}; \node (v1) at (-1,-1) {$v_1$}; \node (v2) at (1,-1) {$v_2$}; \node
      (u) at (0,-2) {$u$}; \draw[thick,->] (t) -- (v1); \draw[thick,->] (t) --
      (v2); \draw[thick,->] (v1) -- (u); \draw[thick,->] (v2) -- (u);
    \end{tikzpicture}
  \item $\to_R$ es \emph{Church-Rosser} o \emph{confluente} si $\to^*_R$
    satisface la propiedad del diamante.
  \item $\to_R$ tiene \emph{formas normales únicas} si $t\to^*_R v_1$ y
    $t\to^*_R v_2$ para términos en forma normal $v_1$ y $v_2$ implica
    $v_1=v_2$.
  \end{itemize}
\end{definicion}

\begin{lema}{3.5}
  ~
  \begin{enumerate}
  \item Si $\to_R$ satisface la propiedad del diamante, entonces es
    Church-Rosser.
  \item Si $\to_R$ es Church-Rosser, entonces tiene formas normales únicas.
  \end{enumerate}
\end{lema}

\begin{definicion}{3.8}
  Llamamos \emph{redex} a un subtérmino de un término que puede reducir.
\end{definicion}

\subsection*{Reducción débil}

\begin{definicion}{3.9}
  Una estrategia de reducción es \emph{débil} si no reduce nunca el cuerpo de
  una función, es decir, si no reduce bajo $\lambda$.
\end{definicion}

\subsection*{Call-by-name}

\begin{definicion}{3.10}
  La estrategia \emph{call-by-name} reduce siempre el \emph{redex} más a la
  izquierda. En caso de ser además débil, será el más a la izquierda que no esté
  debajo de un $\lambda$.
\end{definicion}

\begin{teorema}{3.11}
  [Estandarización] Si un término reduce a un término en forma normal, entonces
  la estrategia \emph{call-by-name} termina.
\end{teorema}

\subsection*{Call-by-value}

\begin{definicion}{3.12}
  A los términos $t$ de PCF tales que $\FV(t)=\emptyset$ y que $t$ esté en forma
  normal, se les llaman \emph{valores}.
\end{definicion}

\begin{definicion}{3.13}
  La estrategia \emph{call-by-value} consiste en evaluar siempre los argumentos
  antes de pasarlos a la función. La idea es que $(\fun xt)v$
  reduce sólo cuando $v$ esté en forma normal. Si la estrategia es débil, y sólo reducimos
  términos cerrados, $v$ es un valor.
\end{definicion}

\newpage
\section*{PCF tipado}

\subsection*{Gramática}

\begin{align*}
  A &::= \nat~|~A\Rightarrow A\\
  t &::= x~|~\fun{x:A}t~|~tt~|~n\in\mathbb N~|~t\otimes t~|~\ifz ttt~|~\fix{x:A}t~|~\letl{x:A}tt
\end{align*}

\paragraph{Contextos}
Un \emph{contexto} nos da tipos para variables, entonces,
en vez de decir $\fun{x:\nat}{yx}:\nat\Rightarrow\nat$, decimos, si
$y:\nat\Rightarrow\nat$, entonces $\fun{x:\nat}{yx}:\nat\Rightarrow\nat$. La
notación que usamos es la siguiente:
\[
  \underbrace{y:\nat\Rightarrow\nat}_{\textrm{contexto}}\vdash\fun{x:\nat}{yx}:\nat\Rightarrow\nat
\]
Genéricamente, queremos definir la relación $\Gamma\vdash t:A$ que asocia un
término $t$ y un contexto $\Gamma$ a un tipo $A$.

\begin{definicion}{4.2}
  \label{def:ST}
  La relación de tipado $\Gamma\vdash t:A$ se define inductivamente por:
  \[\arraycolsep=10pt\def\arraystretch{2.6}
    \begin{array}{cc}
      \infer[ax_v]
      {\Gamma,x:A\vdash x:A}
      {}
      &
        \infer[ax_c]
        {\Gamma\vdash n:\nat}
        {}
      \\
      \infer[\Rightarrow_i]
      {\Gamma\vdash\fun{x:A}t:A\Rightarrow B}
      {\Gamma,x:A\vdash t:B}
      &
        \infer[\Rightarrow_e]
        {\Gamma\vdash tu:B}
        {\Gamma\vdash t:A\Rightarrow B & \Gamma\vdash u:A}
      \\
      \infer[\otimes]
      {\Gamma\vdash t\otimes u:\nat}
      {\Gamma\vdash t:\nat & \Gamma\vdash u:\nat}
      &
        \infer[\mathsf{ifz}]
        {\Gamma\vdash\ifz tuv:A}
        {\Gamma\vdash t:\nat & \Gamma\vdash u:A & \Gamma\vdash v:A}
      \\
      \infer[\mathsf{fix}]
      {\Gamma\vdash\fix{x:A}t:A}
      {\Gamma,x:A\vdash t:A}
      &
        \infer[\mathsf{let}]
        {\Gamma\vdash\letl{x:A}ut:B}
        {\Gamma,x:A\vdash t:B & \Gamma\vdash u:A}
    \end{array}
  \]
  Donde $\otimes$ son cuatro reglas, una para cada operación aritmética.
\end{definicion}

\begin{teorema}{4.4}
  [Subject reduction o conservación de tipos]
  \label{thm:SR}
  ~

  Si $\Gamma\vdash t:A$ y $t\to u$ entonces $\Gamma\vdash u:A$.
\end{teorema}

\begin{teorema}{4.5}
  [Teorema de Tait o normalización fuerte]
  \label{thm:SN}
  ~

  Todo término tipado que no contenga a $\mathsf{fix}$, termina.
\end{teorema}

\newpage
\section*{Inferencia de tipos simples}

\paragraph{Estilo Curry:} $\vdash\fun x{x+1}:\nat\Rightarrow\nat$

\paragraph{Gramática}
\begin{align*}
  A &::= \nat~|~A\Rightarrow A\\
  t &::= x~|~\fun xt~|~tt~|~n~|~t\otimes t~|~\ifz ttt~|~\fix xt~|~\letl xtt
\end{align*}

\paragraph{Reglas de tipado}
\[\arraycolsep=10pt\def\arraystretch{2.6}
  \begin{array}{cc}
    \infer[ax_v]{\Gamma,x:A\vdash x:A}{} &
    \infer[ax_c]{\Gamma\vdash n:\nat}{} \\
    \infer[\Rightarrow_i]
      {\Gamma\vdash\fun xt:A\Rightarrow B}
      {\Gamma,x:A\vdash t:B} &
    \infer[\Rightarrow_e]
      {\Gamma\vdash tu:B}
      {\Gamma\vdash t:A\Rightarrow B & \Gamma\vdash u:A} \\
    \infer[\otimes]
      {\Gamma\vdash t\otimes u:\nat}
      {\Gamma\vdash t:\nat & \Gamma\vdash u:\nat} &
    \infer[\mathsf{ifz}]
      {\Gamma\vdash\ifz tuv:A}
      {\Gamma\vdash t:\nat & \Gamma\vdash u:A & \Gamma\vdash v:A} \\
    \infer[\mathsf{fix}]
      {\Gamma\vdash\fix xt:A}
      {\Gamma,x:A\vdash t:A} &
    \infer[\mathsf{let}]
      {\Gamma\vdash\letl xut:B}
      {\Gamma,x:A\vdash t:B & \Gamma\vdash u:A}
  \end{array}
\]

\begin{definicion}{5.1}
  $A(X) = $ un tipo cualquiera que contiene alguna variable $X$. \\
  $\theta = $ una substitución de meta-variables por tipos.
\end{definicion}

\begin{teorema}{5.3}
  $\vdash t:A(X) \Rightarrow \ \vdash t:\theta A(X)$ para cualquier
  substitución $\theta$.
\end{teorema}

\clearpage
\subsection*{Algoritmo de Hindley}

\[\arraycolsep=10pt\def\arraystretch{2.6}
  \begin{array}{cc}
  \infer[hn]{\Gamma\vdash n\rightsquigarrow\nat,\emptyset}{} \\
  \infer[hx]{\Gamma,x:X\vdash x\rightsquigarrow X,\emptyset}{} \\
  \infer[h\lambda]
      {\Gamma\vdash\fun xt\rightsquigarrow A\Rightarrow B,\tau}
      {\Gamma,x:A\vdash t\rightsquigarrow B,\tau} \\
  \infer[hX]
      {\Gamma\vdash tu\rightsquigarrow X,\tau\cup\sigma\cup\{A=B\Rightarrow X\}}
      {\Gamma\vdash t\rightsquigarrow A,\tau & \Gamma\vdash u\rightsquigarrow B,\sigma } \\
  \infer[h\otimes]
      {\Gamma\vdash t\otimes u\rightsquigarrow\nat,\tau\cup\sigma\cup\{A=\nat,B=\nat\}}
      {\Gamma\vdash t\rightsquigarrow A,\tau & \Gamma\vdash u\rightsquigarrow B,\sigma } \\
  \infer[h\mathsf{ifz}]
      {\Gamma\vdash\ifz tuv\rightsquigarrow A,\tau\cup\sigma\cup\rho\cup\{B=\nat,A=C\}}
      {   \Gamma\vdash t\rightsquigarrow B,\tau
        & \Gamma\vdash u\rightsquigarrow A,\sigma
        & \Gamma\vdash v\rightsquigarrow C,\rho } \\
  \infer[h\mathsf{fix}]
      {\Gamma\vdash\fix xt\rightsquigarrow B,\tau\cup\{A=B\}}
      {\Gamma,x:A\vdash t\rightsquigarrow B,\tau} \\
  \infer[h\mathsf{let}]
      {\Gamma\vdash\letl xut\rightsquigarrow A,\tau\cup\sigma\cup\{B=C\}}
      {\Gamma\vdash u\rightsquigarrow C,\tau & \Gamma,x:B\vdash t\rightsquigarrow A,\sigma}
  \end{array}
\]
    
\subsection*{Algoritmo de unificación de Robinson}
\begin{definicion}
  Sea $\theta$ una substitución $A_1/X_1,\dots,A_n/X_n$. Decimos que $\theta$ es
  una \emph{solución} del conjunto de ecuaciones $\tau$ si para todo $B=C$ en
  $\tau$, $\theta B$ y $\theta C$ son idénticos.
\end{definicion}

\begin{teorema}
  Si $\vdash t\rightsquigarrow A,\tau$, entonces para toda solución $\theta$ de
  $\tau$, $\vdash t:\theta A$.

  Más general: si $\Gamma\vdash t\rightsquigarrow A,\tau$, entonces para toda
  solución $\theta$ de $\tau$, $\theta\Gamma\vdash t:\theta A$, donde
  $\theta\Gamma$ es la substitución en cada tipo que aparece en $\Gamma$.
\end{teorema}

\paragraph{Algoritmo}
\begin{enumerate}
  \item Si $\exists$ ecuación $A\Rightarrow B=C\Rightarrow D$ \emph{reemplazarla}
    por $A=C$ y $B=D$.
  \item Si $\exists$ ecuación $\nat=\nat$, \emph{borrarla}.
  \item Si $\exists$ ecuación $\nat=A\Rightarrow B$, o $A\Rightarrow B=\nat$, \emph{responder error}.
  \item Si $\exists$ ecuación $X=X$, \emph{borrarla}.
  \item Si $\exists$ ecuación $A=X$ o $X=A$ donde $X$ \emph{aparece} en $A$ pero $A\neq X$, \emph{responder error}.
  \item Si $\exists$ ecuación $A=X$ o $X=A$ donde $X$ \emph{no aparece} en $A$ pero
    aparece en otras ecuaciones ecuaciones, substituir $X$ por $A$ en todas las
    ecuaciones.
\end{enumerate}


\newpage
\section*{Polimorfismo}

\begin{gather*}
  \fun xx : \forall X.(X\Rightarrow X) \\
  \Gamma \vdash t : \forall x.A \Rightarrow \Gamma \vdash t: A[B/X]
\end{gather*}

\subsection*{Tipos polimórficos}

\begin{definicion}{6.2}
  Un esquema de tipo tiene forma $\forall X_1\dots\forall X_n.A$,
  con $A$ un tipo y $n\geq 0$.
\end{definicion}

\paragraph*{Gramática a dos niveles}: uno para tipos y otro para
esquemas de tipos.

\begin{align*}
  A & ::= X \mid \nat \mid A\Rightarrow A\\
  \alpha & ::= [A]\mid\forall X.\alpha
\end{align*}

Si $A$ es un tipo, $[A]$ es un esquema de tipo formado por el tipo $A$ donde
ninguna variable está cuantificada.

\begin{definicion}{6.3}
  Extendemos la definición de variables libres (\FV) para tipos:
  \begin{align*}
    \FV([X]) & = \{X\}\\
    \FV([\nat]) & = \emptyset\\
    \FV([A\Rightarrow B]) & = \FV([A])\cup\FV([B])\\
    \FV(\forall X.\alpha) & = \FV(\alpha)\setminus\{X\}
  \end{align*}
\end{definicion}

\begin{definicion}{6.4}
  El sistema de tipos asocia contextos y términos con esquemas de tipos,
  $\Gamma\vdash t:\alpha$, y viene dado por
  \[
    \begin{array}{c@{\qquad\qquad}c}
      \infer[ax_v]{\Gamma,x:\alpha\vdash x:\alpha}{}
      &
        \infer[ax_c]{\Gamma\vdash n:[\nat]}{}
      \\[2ex]
      \infer[\Rightarrow_i]{\Gamma\vdash\fun xt:[A\Rightarrow B]}
      {\Gamma,x:[A]\vdash t:[B]}
      &
        \infer[\Rightarrow_e]{\Gamma\vdash tu:[B]}
        {
        \Gamma\vdash t:[A\Rightarrow B]
      &
        \Gamma\vdash u:[A]
        }
      \\[2ex]
      \infer[\otimes]{\Gamma\vdash t\otimes u:[\nat]}
      {\Gamma\vdash t:[\nat] & \Gamma\vdash u:[\nat]}
      &
        \infer[\mathsf{ifz}]{\Gamma\vdash\ifz tuv:[A]}
        {\Gamma\vdash t:[\nat] & \Gamma\vdash u:[A] & \Gamma\vdash v:[A]}
      \\[2ex]
      \infer[\mathsf{fix}]{\Gamma\vdash\fix xt:[A]}{\Gamma,x:[A]\vdash t:[A]}
      &
        \infer[\mathsf{let}]{\Gamma\vdash\letl xtu:[A]}
        {\Gamma\vdash t:\alpha & \Gamma,x:\alpha\vdash u:[A]}
      \\[2ex]
      \textrm{Si }X\notin\FV(\Gamma),\ \vcenter{\infer[\forall_i]{\Gamma\vdash t:\forall X.\alpha}{\Gamma\vdash t:\alpha}}
      &
        \vcenter{\infer[\forall_e]{\Gamma\vdash t:\alpha[A/X]}{\Gamma\vdash t:\forall X.\alpha}}
    \end{array}
  \]
\end{definicion}

\paragraph*{Sistema dirigido por sintaxis de Hindley-Milner}

Primero definimos la siguiente relación entre tipos.
\[
  \alpha\preceq\forall X.\alpha\textrm{, si }X\notin\FV(\alpha)
  \qquad\textrm{y}\qquad \forall X.\alpha\preceq\alpha[A/X]
\]
El sistema es el siguiente:
\[
  \infer{\Gamma,x:\alpha\vdash x:\beta}{\alpha\preceq\beta} \qquad
  \infer{\Gamma\vdash n:[\nat]}{} \qquad \infer{\Gamma\vdash\lambda
    x.t:[A\Rightarrow B]}{\Gamma,x:[A]\vdash t:[B]} \qquad \infer{\Gamma\vdash
    tu:[B]}{\Gamma\vdash t:[A\Rightarrow B] & \Gamma\vdash u:[A]}
\]
\[
  \infer{\Gamma\vdash t\otimes u:[\nat]}{\Gamma\vdash t:[\nat] & \Gamma\vdash
    u:[\nat]} \qquad \infer{\Gamma\vdash\ifz tuv:[A]}{\Gamma\vdash t:[\nat] &
    \Gamma\vdash u:[A] &\Gamma\vdash v:[A]}
\]
\newcommand\cuad[1]{[#1]}
\[
  \infer{\Gamma\vdash\fix xt:[A]}{\Gamma,x:[A]\vdash t:[A]} \qquad
  \infer[\begin{array}{l}\vec\Gamma(A)=\forall\vec X.\cuad A,\\ \vec
    X=\FV(A)\setminus\FV(\Gamma)\end{array}]{\Gamma\vdash\letl
    xut:[B]}{\Gamma\vdash u:[A] & \Gamma,x:\vec\Gamma(A)\vdash t:[B]}
\]

\paragraph*{System F}

\[
  A::= \nat\mid A\Rightarrow A\mid X\mid \forall X.A
\]

\[\arraycolsep=10pt\def\arraystretch{2.6}
    \begin{array}{cc}
      \infer[ax_v]
      {\Gamma,x:A\vdash x:A}
      {}
      &
        \infer[ax_c]
        {\Gamma\vdash n:\nat}
        {}
      \\
      \infer[\Rightarrow_i]
      {\Gamma\vdash\fun{x:A}t:A\Rightarrow B}
      {\Gamma,x:A\vdash t:B}
      &
        \infer[\Rightarrow_e]
        {\Gamma\vdash tu:B}
        {\Gamma\vdash t:A\Rightarrow B & \Gamma\vdash u:A}
      \\
      \infer[\otimes]
      {\Gamma\vdash t\otimes u:\nat}
      {\Gamma\vdash t:\nat & \Gamma\vdash u:\nat}
      &
        \infer[\mathsf{ifz}]
        {\Gamma\vdash\ifz tuv:A}
        {\Gamma\vdash t:\nat & \Gamma\vdash u:A & \Gamma\vdash v:A}
      \\
      \infer[\mathsf{fix}]
      {\Gamma\vdash\fix{x:A}t:A}
      {\Gamma,x:A\vdash t:A}
      &
        \infer[\mathsf{let}]
        {\Gamma\vdash\letl{x:A}ut:B}
        {\Gamma,x:A\vdash t:B & \Gamma\vdash u:A}
      \\
      \infer[\forall_i]{\Gamma\vdash t:\forall X.A}{\Gamma\vdash t:A & X\notin\FV(\Gamma)}
      &
      \infer[\forall_e]{\Gamma\vdash t:A[B/X]}{\Gamma\vdash t:\forall X.A}
    \end{array}
  \]

\newpage
\section*{Ejecución}

\subsection*{Interprete CBN}

\begin{equation*}
  \infer[x\notin\mathsf D(\Delta)]
    {\Gamma,x=\thunk {\bm{t}}{\Gamma'},\Delta\vdash x\hra v}
    {\Gamma'\vdash \bm{t}\hra v}
  \qquad
  \infer[]
    {\Gamma\vdash n\hra n}
    {}
  \qquad
  \infer[\textrm{Si }n\otimes m=p]
    {\Gamma\vdash \bm{t} \otimes \bm{u} \hra p}
    {\Gamma\vdash \bm{t} \hra n & \Gamma\vdash \bm{u}\hra m }
\end{equation*}

\begin{equation*}
  \infer[]
    {\Gamma\vdash\fun xt\hra\thunk {\fun xt}\Gamma}
    {}
  \qquad
  \infer[]
    {\Gamma\vdash \bm{t} \bm{r} \hra v}
    {
      \Gamma\vdash \bm{t} \hra \thunk{\fun x{t'}}{\Gamma'}
      & \Gamma',x=\thunk{\bm{r}}{\Gamma} \vdash t'\hra v
    }
\end{equation*}

\begin{equation*}
  \infer
    {\Gamma\vdash \ifz {\bm t}{\bm r}{s} \hra v}
    {\Gamma\vdash \bm t \hra 0 & \Gamma \vdash \bm r \hra v}
  \qquad
  \infer[\textrm{Si }n\neq 0]
    {\Gamma\vdash \ifz {\bm t}{r}{\bm s} \hra v}
    {\Gamma\vdash \bm t \hra n & \Gamma\vdash \bm s \hra v}
\end{equation*}

\begin{equation*}
  \infer
    {\Gamma \vdash \letl {x}{\bm r}{\bm t} \hra v}
    {\Gamma,x=\thunk{\bm r}{\Gamma} \vdash \bm t \hra v}
  \qquad
    \infer{\Gamma\vdash\fix{x}{\bm t} \hra v}
    {\Gamma,x=\thunk{\fix{x}{\bm t}}\Gamma\vdash \bm t \hra v}
\end{equation*}

\subsection*{Intérprete en CBV}

\begin{equation*}
  \infer[x\notin\mathsf D(\Delta)]
    {\Gamma,x=v,\Delta\vdash x\hra v}
    {}
  \qquad
  \infer[x\notin\mathsf D(\Delta)]
    {\Gamma,x=\thunk{\bm{\fix yt}}{\bm{\Gamma'}},\Delta\vdash x\hra v}
    {\bm{\Gamma'} \vdash \bm{\fix yt} \to v}
\end{equation*}

\begin{equation*}
  \infer[]
    {\Gamma\vdash n \hra n}
    {}
  \qquad
  \infer[\textrm{Si }n\odot m=p]
    {\Gamma\vdash \bm t \odot \bm u \hra p}
    {\Gamma\vdash \bm t \hra n & \Gamma\vdash \bm u \hra m }
\end{equation*}

\begin{equation*}
  \infer[]
    {\Gamma\vdash\fun xt\hra\thunk {\fun xt}\Gamma}
    {}
  \qquad
  \infer
    {\Gamma\vdash \bm{t} \bm{r} \hra v}
    {
      \Gamma\vdash \bm r \hra w
      &\Gamma\vdash \bm t \hra\thunk {\fun x{t'}}{\Gamma'}
      &\Gamma',x=w\vdash t'\hra v
    }
\end{equation*}

\begin{equation*}
  \infer[]
    {\Gamma\vdash \ifz{\bm t}{\bm r}{s}\hra v}
    {
      \Gamma\vdash  \bm t \hra 0
      &\Gamma\vdash \bm r \hra v
    }
    \qquad
    \infer[\textrm{Si }n\neq 0]
      {\Gamma\vdash \ifz{\bm t}{r}{\bm s} \hra v}
      {
        \Gamma\vdash  \bm t \hra n
        &\Gamma\vdash \bm s \hra v
      }
\end{equation*}

\begin{equation*}
  \infer[]
    {\Gamma\vdash\letl{x}{\bm r}{\bm t} \hra v}
    {
      \Gamma\vdash \bm r \hra w
      &\Gamma,x=w\vdash \bm t \hra v
    }
    \qquad
    \infer[]
      {\Gamma \vdash \fix x{\bm t} \hra v}
      {\Gamma,x=\thunk{\fix x{\bm t}}\Gamma\vdash \bm t \hra v}
\end{equation*}

\newpage
\subsection*{Compilador}

\subsubsection*{Semántica Operacional ``Assembler''}

\begin{align*}
  (a, s, e, \Mkclos i; c)                   & \to (\thunk ie, s, e, c)    \\
  (a, s, e, \Push; c)                       & \to (a, a{\append}s, e, c)  \\
  (a, s, e, \Extend; c)                     & \to (a, s, e\concat [a], c) \\
  (a, s, e, \Search n; c)                   & \to (v, s, e, c)   \qquad\qquad v = e[n] \textrm{ desde la derecha}\\
  (a, s, e, \Pushenv; c)                    & \to (a, e{\append}s, e, c)  \\
  (a, e'{\append}s, e, \Popenv; c)          & \to (a, s, e',c)   \\
  (\thunk i{e'}, w{\append}s, e, \Apply; c) & \to (\thunk i{e'}, s, e'\concat [\thunk i{e'},w], i; c)\\
  (a, s, e, \Ldi n; c)                      & \to (n, s, e, c)    \\
  (n, m{\append}s, e, \Add; c)              & \to (n+m, s, e, c)  \\
  (n, m; s, e, \Sub; c)                     & \to (n-m, s, e, c)  \\
  (n, m; s, e, \Mult; c)                    & \to (n*m, s, e, c)  \\
  (n, m; s, e, \Div; c)                     & \to (n/m, s, e, c)  \\
  (0, s, e, \Test i j; c)                   & \to (0, s, e, i; c) \\
  (n, s, e, \Test i j; c)                   & \to (n, s, e,j; c) \qquad\qquad \textrm{Si }n>0 
\end{align*}

\subsubsection*{Compilador PCF}

\begin{align*}
  |x|_e           &= \Search n \qquad\qquad\textrm{($n$ = \s{index}$(e, x)$ desde la derecha)}\\
  |tu|_e          &= \Pushenv;  |u|_e; \Push;  |t|_e;  \Apply;  \Popenv\\
  |\fun xt|_e     &= \Mkclos {|t|_{e\concats[\_,x]}} \\
  |\fixfun fxt|_e &= \Mkclos {|t|_{e\concats[f,x]}}  \\
  |n|_e           &= \Ldi n \\
  |t+u|_e         &= |u|_e; \Push; |t|_e; \Add\\
  |t-u|_e         &= |u|_e; \Push; |t|_e; \Sub\\
  |t*u|_e         &= |u|_e; \Push; |t|_e; \Mult\\
  |t / u|_e       &= |u|_e; \Push; |t|_e; \Div\\
  |\ifz tuv|_e    &= |t|_e; \Test {|u|_e} {|v|_e}\\
  |\letl xtu|_e   &= \Pushenv;  |t|_e;  \Extend;  |u|_{e\concats [x]};  \Popenv
\end{align*}

\newpage
\subsubsection*{Descripción de Instrucciones ``Assembler''}
\begin{tabular}{lp{12.6cm}}\\
  \hline $\Ldi n$    & Pone $n$ en el acumulador \\
  \hline $\Push$     & Pone el contenido del acumulador en la pila \\
  \hline $\Add$      & Suma el tope de la pila con el acumulador, pone el resultado en el acumulador y borra el elemento de la pila \\
  \hline $\Sub$      & Idem restando \\
  \hline $\Mult$     & Idem multiplicando \\
  \hline $\Div$      & Idem dividiendo \\
  \hline $\Extend$   & Extiende el entorno agregando el contenido del acumulador en la última posición \\
  \hline $\Search n$ & Busca el valor de la posición $n$ en el entorno y lo pone en el acumulador \\
  \hline $\Pushenv$  & Mete el contenido del entorno al tope de la pila \\
  \hline $\Popenv$   & Saca el tope de la pila a un entorno \\
  \hline $\Mkclos i$ & Mete el $\thunk ie$ en el acumulador ($e$ es el entorno actual) \\
  \hline $\Apply$    & Toma un cierre $\thunk ie$ del acumulador
  y pone el entorno $e, \thunk ie, W$, donde $W$ es el elemento de
  arriba de la pila; retira el elemento de arriba de la pila, y pone el
  resto de las instrucciones $i$ en el registro de código \\
  \hline $\Test ij$  & Mete $i$ o $j$ en el acumulador dependiendo de si el acumulador es $0$ o $n>0$ \\ 
  \hline
\end{tabular}
\newpage
\section*{Registros y Objetos}

\subsection*{PCF con Registros}

Sea $L$ in conjunto infinito de etiquetas válidas donde $l,l_1,\dots,l_n\in L$.

\begin{equation*}
  t ::= \dots~|~\{l_1=t,l_2=t,\dots,l_n=t\}~|~t\cdot l~|~t(l\leftarrow t)
\end{equation*}

\paragraph{Semántica operacional}

\begin{align*}
  \{l_1=t_1,\dots,l_n=t_n\}\cdot l_i &\to t_i\\
  \{l_1=t_1,\dots,l_n=t_n\}(l_i\leftarrow u) &\to
  \{l_1=t_1,\dots,l_{i-1}=t_{i-1},l_i=u,l_{i+1}=t_{i+1},\dots,l_n=t_n\}
\end{align*}

\subsection*{PCF con Objetos}

\begin{align*}
  t ::= \dots~|~\{l_1=\sigma s.t,l_2=\sigma s.t,\dots,l_n=\sigma s.t\}~|~t\#l~|~t(l\leftarrow \sigma s.t)
\end{align*}

donde $l,l_1,\dots,l_n\in L$, $s\in\mathsf{Vars}$.

\paragraph{Semántica operacional}

\begin{align*}
  \{l_1 = \sigma s.t_1,\dots,l_n = \sigma s.t_n\}\#l_i
    &\to t_i[\{l_1 = \sigma s.t_1,\dots,l_n = \sigma s.t_n\}/s_i] \\
  \{l_1 = \sigma s.t_1,\dots,l_n = \sigma s.t_n\}(l_i\leftarrow\sigma s.u_i)
    &\to \{l_1 = \sigma s.t_1,\dots,l_{i-1} = \sigma s.t_{i-1}, \\
    &~\qquad l_i = \sigma s.u_i, \\
    &~\qquad l_{i+1} = \sigma s.t_{i+1},\dots,l_n = \sigma s.t_n\}
\end{align*}

\section*{Semántica denotacional de PCF tipado}

\paragraph{Interpretación de los tipos}

\begin{align*}
  \sem{\nat}           &= \mathbb N \\
  \sem{A\Rightarrow B} &= \sem A\to\sem B \\
  \semT x              &= \theta(x) \\
  \semT{\fun{x:A}t}    &=\funSem aA{\semT[\theta,x=a] t}\\
  \semT{tr}            &=\semT t\semT r\\
  \semT n              &= n\\
  \semT{t\otimes r}    &=
  \semT t\otimes\semT r \\
  \semT{\ifz trs}      &=
    \left\{\begin{array}{ll}
      \semT r & \textrm{si }\semT t=0\\
      \semT s & \textrm{si }\semT t\in\mathbb N^*\\
      \bot_A  & \textrm{si }\semT t=\bot_{\mathbb N}\textrm{ y $A$ es el tipo del término}
    \end{array}\right. \\
  \semT{\letl{x:A}rt}  &=\semT[\theta,x=\semT r] t\\
  \semT{\fix{x:A}t}    &= \mathsf{FIX}(\funSem aA{\semT[\theta,x=a] t})
\end{align*}
donde $\mathsf{FIX}(f)$ es el mínimo punto fijo de $f$.

\end{document}

